\subsection{\ppp}
\subsubsection{Wstęp}
\ppp (Point-to-Point Protocol) to rozwiązanie służące do przenoszenia datagramów IP.
Najczęściej jest stosowany do tworzenia połączeń opartych na połączeniach szeregowych (takich jak modemy, łącza optyczne)\cite{ppp:stevens-ppp}.

W przeciwieństwie do \arp{} \ppp{} należy traktować raczej jako zbiór protokołów.
Protokoły te pozwalają między innymi na utworzenie połączenia \ppp (LCP), i konfiguracje połączenia, które będzie korzystać z połączenia \ppp (NCP) -- np. połączenia IP.

Zakresem działania \textbf{LCP} (\emph{Link Control Protocols}) jest samo zestawienie połączenia pomiędzy dwoma urządzeniami.
Wymagania wobec samego połączenia są minimalne w stosunku do innych protokołów, wystarczy że wspiera komunikację dwukierunkową.

Zwykle samo \ppp nie koncentruje się na zapewnieniu niezawodności połączenia i retransmisji danych, choć istnieją standardy i protokoły, które mogą rozwiązywać problemy z niezawodnością sieci w ramach \ppp{}.
To zadanie jest delegowane do protokółu, który działa wewnątrz połączenia \ppp{}, np. \ip{}.

Poza tym istnieją też protokoły i rozszerzenia \ppp{} wspierające szyfrowanie, kompresję i raportowanie stanu połączenia.

Ponieważ nasza stacja robocza na laboratorium nie była wyposażona w modem, w ramach ćwiczenia \ppp{} wykonaliśmy połączenie \tcp{} poprzez \pppoe{}.

Zwykle połączenia tego typu stosowane są do połączenia z modemem lub podobnym urządzeniem poprzez sieć LAN lub WAN \cite{ppp:stevens-pppoe} (ostatecznie na łączu ethernetowym można zestawić połączenie \tcp{} nie korzystając z \ppp{}).

\subsubsection{Przygotowanie ćwiczenia}

Konfiguracja sprzętowa połączenia przedstawiała się następująco:

\begin{figure}[h!]
  \centering
  \includegraphics[width=11cm]{figury/ppp/schemat-przed.pdf}
  \caption{Schemat sieci - połączenie PPP.}
  \label{fig:ppp:schemat-sieci-przed}
\end{figure}

\pppcli{} i \pppserv{} połączyliśmy skrosowanym kablem.

Zaczęliśmy od zdekonfigurowania interfejsów \emph{sis2} i \emph{fxp0}.
Następnie podnieśliśmy te interfejsy.

Maszyna \pppserv{} pełniła rolę serwera PPP, a maszyna \pppcli{} - klienta PPP.
Na maszynie \pppcli{} skonfigurowaliśmy połączenie w taki sposób, aby wynegocjowane zostało następujące połączenie \tcp{}:

\begin{figure}[h!]
  \centering
  \includegraphics[width=11cm]{figury/ppp/schemat-po.pdf}
  \caption{Schemat sieci tunelowanej przez \ppp{}.}
  \label{fig:ppp:schemat-sieci-przed}
\end{figure}

\subsubsection{Wykonanie ćwiczenia}
Skonfigurowaliśmy klienta PPP na maszynie \pppcli{} w następujący sposób\footnote{/etc/ppp/ppp.conf}:

\begin{lstlisting}
default:

  # logowanie

  set log Phase tun command

polaczenie:

  # docelowa konfiguracja sieci

  set ifaddr 10.0.0.1/0 10.0.0.2/0 

  # urządzenie, z którego należy korzystać (fxp0)
  # nie udało nam się połączyć bez ':*', które
  # oznacza, że klient powinien próbować połączenie
  # z dowolnym providerem

  set device PPPoE:fxp0:* #1

  set authname ppp
  set authkey ppp
  set dial
  set login
  add default HISADDR
\end{lstlisting}

Następnie na \pppserv{} uruchomiliśmy proces pppoed:


\begin{lstlisting}
  # service pppoed onestart sis2
\end{lstlisting}

I na \pppcli{} wykonaliśmy następującą komendę, która zestawiła połączenie:


\begin{lstlisting}
  $ ppp -ddial polaczenie
\end{lstlisting}

Wynik polecenia \texttt{ifconfig} następnie wskazuje, że mamy zestawione połączenie.


\begin{lstlisting}
tun0: flags=8051<UP,POINTOPOINT,RUNNING,MULTICAST> metric 0 mtu 1492
        options=80000<LINKSTATE>
        inet 10.0.0.2 --> 10.0.0.1 netmask 255.255.255.255
        Opened by PID 27226
\end{lstlisting}

\subsubsection{Obserwacja połączenia i wnioski}
Udało nam się zalogować negocjację połączenia przez protokół \texttt{LCP}
\footnote{maszyna \pppserv{} nazywała się wtedy \texttt{so5501b}}
:

\begin{lstlisting}
17:36:01.655628 PPPoE PADI [Host-Uniq 0xC0B9C5C2] [Service-Name "*"]
17:36:34.096172 PPPoE PADI [Host-Uniq 0x803AD9C2] [Service-Name "*"]
17:36:34.102227 PPPoE PADO [AC-Name "so5501b"] [Service-Name "*"] 
    [Host-Uniq 0x803AD9C2] [AC-Cookie 0x400818C3]
17:36:34.102367 PPPoE PADR [Host-Uniq 0x803AD9C2] 
    [AC-Cookie 0x400818C3] [AC-Name "so5501b"] [Service-Name "*"]
17:36:34.102455 PPPoE PADS [ses 0x1] [AC-Name "so5501b"] 
    [Service-Name "*"] [Host-Uniq 0x803AD9C2] [AC-Cookie 0x400818C3]
17:36:35.365394 PPPoE [ses 0x1] LCP, Conf-Request (0x01), id 1, length 26
17:36:35.365863 PPPoE [ses 0x1] LCP, Conf-Request (0x01), id 1, length 26
17:36:35.365913 PPPoE [ses 0x1] LCP, Conf-Ack (0x02), id 1, length 26
17:36:35.366971 PPPoE [ses 0x1] LCP, Conf-Ack (0x02), id 1, length 26
\end{lstlisting}

Jest to zgodne z tym, jak taka negocjacja powinna wyglądać \cite{ppp:stevens-pppoe}:

\begin{figure}[h!]
  \centering
  \includegraphics[width=5cm]{figury/ppp/padi-pado.pdf}
  \caption{Negocjacja połączenia \ppp.}
  \label{fig:ppp:schemat-sieci-przed}
\end{figure}

Po ustanowieniu sesji widzimy, że protokół LCP zaczyna konfigurować połączenie \tcp{} zgodnie z konfiguracją w pliku \texttt{/etc/ppp/ppp.conf}.

Za pomocą poleceń \texttt{ping} i \texttt{netcat} sprawdziliśmy, że połączenie faktycznie działa.
\begin{lstlisting}
17:40:27.658383 PPPoE [ses 0x3] IP so4801c.iem.pw.edu.pl > 10.0.0.1: ICMP echo request, id 61802, seq 0, length 64
17:40:27.660076 PPPoE [ses 0x3] IP 10.0.0.1 > so4801c.iem.pw.edu.pl: ICMP echo reply, id 61802, seq 0, length 64
17:40:28.657813 PPPoE [ses 0x3] compressed PPP data
17:40:28.658171 PPPoE [ses 0x3] compressed PPP data
\end{lstlisting}
Widać tutaj, że zestawiliśmy połączenie z kompresją - przynajmniej wg. programu \texttt{tcpdump}.
Widzimy też, że możemy obserwować pakiety zarówno przed jak i po opakowaniu w ramki PPP (choć \texttt{tcpdump} nie pozwala nam zobaczyć, co dokładnie jest w pakietach przenoszonych przez \texttt{PPP}.
